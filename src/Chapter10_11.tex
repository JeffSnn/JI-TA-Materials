%----------------------------------------------------------------------------------------
%	TITLE PAGE
%----------------------------------------------------------------------------------------
\documentclass[aspectratio=169]{beamer}
% use 16:9 in new slides!

\usepackage{tikz}
\usepackage{array}

\usepackage[skins]{tcolorbox}

\def\blankline{\\[12pt]}
\def\bpr#1#2{
\begin{tcolorbox}
[boxsep=.15cm,left=.2cm,right=.2cm,oversize,boxrule=0mm,
colback=green!55!blue!20!white!60,
colframe=red!50!yellow!50!white,
colbacktitle=blue!50, 
coltitle=black,enhanced,drop fuzzy shadow,
fonttitle=\bfseries,title=#1]
#2
\end{tcolorbox}
}

\def\bdef#1#2{
\begin{tcolorbox}
[boxsep=.15cm,left=.2cm,right=.2cm,oversize,boxrule=0mm,
colback=white!60,
colframe=red!50!yellow!50!white,
colbacktitle=green!50!yellow!60!gray, 
coltitle=black,enhanced,drop fuzzy shadow,
fonttitle=\bfseries,title=#1]
#2
\end{tcolorbox}
}

\def\bthe#1#2{
\begin{tcolorbox}
[boxsep=.15cm,left=.2cm,right=.2cm,oversize,boxrule=0mm,
colback=white!60,
colframe=red!50!yellow!50!white,
colbacktitle=red!52, 
coltitle=black,enhanced,drop fuzzy shadow,
fonttitle=\bfseries,title=#1]
#2
\end{tcolorbox}
}

\def\tcb#1{
\begin{tcolorbox}
[boxsep=.15cm,left=.2cm,right=.2cm,oversize,boxrule=0mm,
colback=white!60,
colframe=red!50!yellow!50!white,
colbacktitle=red!50!yellow!50!white, coltitle=black,enhanced,drop fuzzy shadow,]
#1
\end{tcolorbox}
}



\definecolor{UMBlue}{RGB}{5,32,103}
\definecolor{UMYellow}{RGB}{255,216,0}
%\usetheme{Madrid}
\setbeamertemplate{headline}{
  \leavevmode%
  \begin{minipage}{0.75\paperwidth}
  \vspace{1.2ex}\hspace{-0.245\paperwidth}
  \resizebox{\paperwidth}{3ex}{
    \tikz{
      \fill [color=UMBlue] (0,0) rectangle (10, 0.13);
      \fill [color=UMYellow] (0,-0.1) rectangle (10, -0.18);
    }
  }
  \begin{beamercolorbox}[wd=\linewidth,ht=2.5ex,dp=1.125ex]{section}
    \insertsubsectionnavigationhorizontal{\linewidth}{}{Slide \insertpagenumber}
  \end{beamercolorbox}
  \end{minipage}
  \begin{minipage}{0.23\paperwidth}
  \hspace{0.2em}
  \includegraphics[width=\linewidth]{assets/logo.png}
  \end{minipage}
}

\setbeamercolor{title}{fg=UMBlue}
\setbeamercolor{frametitle}{fg=UMBlue}
\setbeamercolor{structure}{fg=UMBlue}

\title[Course number]{VC210 Chemical Equilibrium and Acid-Base} 
\author[]{Sun Qizhen}
\institute[UMJI-SJTU]
{
	University of Michigan - Shanghai Jiaotong University
	\\\medskip
	Joint Institute
}
%----------------------------------------------------------------------------------------
%	Highlight the title of the current section
%----------------------------------------------------------------------------------------
\AtBeginSection[]
{
  \begin{frame}
    \frametitle{Table of Contents}
    \tableofcontents[currentsection]
  \end{frame}
}



\begin{document}
% insert title page---------------------------
\maketitle
%insert contents------------------------------
\begin{frame}
  \frametitle{Table of Contents}
  \tableofcontents
\end{frame}


\section{Chemical Equilibrium}
\begin{frame}
  \frametitle{Chemical Equilibrium}
  Chemical equilibrium is a dynamic equilibrium. When the equilibrium is reached, forward rate equals reverse rate.
  \blankline
  For such a reaction 
  \begin{equation*}
    aA + bB \rightleftharpoons cC + dD
  \end{equation*}
  The reaction quotient is defined as
  \begin{equation*}
    Q = \frac{a_C^c a_D^d}{a_A^a a_B^b}
  \end{equation*}
  where $a_i$ is the activity. Usually, $a_i = \frac{p_i}{1\text{ bar}}$ for gases and $a_i = \frac{c_i}{1 \text{ mol/L}}$ for solutions. 
  Under equilibrium, $Q = K$ is called the equilibrium constant.
\end{frame}
\begin{frame}
  \frametitle{Origin of $K$}
  Here is an equation that relates the Gibbs free energy change to reaction quotient.
  \begin{equation*}
    \Delta G_r = \Delta G_r^{\circ} + RT \ln Q
  \end{equation*}
  Under equilibrium, 
  \begin{equation*}
    \Delta G_r = 0 \Rightarrow \Delta G_r^{\circ} = -RT \ln K
  \end{equation*}
  Recall that $\Delta G = 0$ means neither the reaction nor the reverse reaction is spontaneous. \\
  Plug in the definition of Gibbs free energy and we can get the relationship between $K$ and $T$.
  \begin{equation*}
    \ln (\frac{K_2}{K_1}) = \frac{\Delta H^{\circ}}{R}\left(\frac{1}{T_1} - \frac{1}{T_2}\right)
  \end{equation*}
\end{frame}
\begin{frame}
  \frametitle{ICE Table}
  When dealing with problems about equilibriums, a tool often used is ICE table.
  \blankline
  We write down all reactants and products, and record their initial, change, and equilibrium concentrations. 
  Get an equation with $K$ and solve for the unknown(s).
  \begin{table}[h]
    \centering
    \begin{tabular}{|c|c|c|c|}
      \hline
      & Initial & Change & Equilibrium \\
      \hline
      A & $a$ & $-x$ & $a-x$ \\
      \hline
      B & $b$ & $-x$ & $b-x$ \\
      \hline
      C & $0$ & $+x$ & $x$ \\
      \hline
      D & $0$ & $+x$ & $x$ \\
      \hline
    \end{tabular}
  \end{table}
\end{frame}
\begin{frame}
  \frametitle{Approximation}
  The "5\%" approximation is commonly used in Chemistry. It is definitely OK if you do not use it. However, I hope 
  you can see how powerful it is. Here is a simple example.
  \blankline
  For the ionization of a weak monoacid, suppose the initial concentration of $HA$ is $c$, and the equilibrium constant is $K_a$, 
  calculate the pH value of the solution.
  \pause
  \blankline
  It is easy to see
  \begin{equation*}
    K_a = \frac{[H^+][A^-]}{[HA]} = \frac{x^2}{c-x}
  \end{equation*}
  Since $x$ is insignificant compared to $c$, we treat $c-x\approx c$ and get
  \begin{equation*}
    [H^+] = x = \sqrt{cK_a}
  \end{equation*}
  In a complicated system, the approximation will greatly simplify the calculation!
\end{frame}
\begin{frame}
  \frametitle{Approximation}
  This slide is not required in this course, but it will help you understand the approximation better.
  \blankline
  A common problem is about the precondition of applying approximation. We can interpret it 
  mathematically. Which ones of the following are correct?
  \begin{itemize}
    \item $x + o(1) = x$
    \item $x + o(x) = x$
    \item $(x + o(x))f(x) = xf(x)$
    \item $(x + o(x))f(x) + g(x) = xf(x) + g(x)$
  \end{itemize}
  In Chemistry, our criteria is that the relative error of the final result is no larger than $5\%$. 
  In most real-life cases, the relative error is very small (the case in the last slide).
\end{frame}
\begin{frame}
  \frametitle{Le Chatelier's principle}
  Le Chatelier's principle describes the response of equilibria to changes in 
  different conditions.
  \blankline
  Here is a concise statement from Tao Te Ching that summarizes the principle.
  \begin{quote}
    Heaven's Way is like stretching a bow.\\
    The high is lowered and the low is raised. \\
    Excess is reduced and deficiency is replenished.
  \end{quote}
  Or simply, the reaction will go in the direction that reduces the effect of the change.
\end{frame}
\begin{frame}
  \frametitle{Exercise}
  The initial partial pressures of nitrogen and hydrogen in a rigid, sealed vessel are 0.010 and 0.020 bar, 
  respectively. The mixture is heated to a temperature at which K = 0.11 for
  \begin{equation*}
    N_2(g) + 3H_2(g) \rightleftharpoons 2NH_3(g)
  \end{equation*}
  Calculate the partial pressure of ammonia at equilibrium.
  \pause
  \blankline
  Ans: $9.26\times 10^{-5}\ bar$. \\
  The ICE table is
  \begin{table}[h]
    \centering
    \begin{tabular}{|c|c|c|c|}
      \hline
      & Initial & Change & Equilibrium \\
      \hline
      $N_2$ & $0.010$ & $-x$ & $0.010-x$ \\
      \hline
      $H_2$ & $0.020$ & $-3x$ & $0.020-3x$ \\
      \hline
      $NH_3$ & $0$ & $+2x$ & $2x$ \\
      \hline
    \end{tabular}
  \end{table}
\end{frame}
\begin{frame}
  \frametitle{Exercise}
  Now, we need to solve the equation
  \begin{equation*}
    \frac{\left(2x\right)^2}{\left(0.01-x\right)\left(0.02-3x\right)^3} = 0.11
  \end{equation*}
  There are some approaches:
  \begin{itemize}
    \item Solve directly with Casio. Note: this approach (Newton's Method) does not always give you the correct answer.
    \item Naive approximation. Solve the function as if $x$ is very small, which gives $2x = \sqrt{0.11\times 0.01\times 0.02^3}\approx 9.38\times 10^{-5}$
    \item Nicer approximation. Treat the equation as the quadratic equation $\frac{(2x)^2}{(0.01 - x)(0.02 - 3\cdot 3x \cdot 0.02^2)} = 0.11$, which gives $2x = 9.26\times 10^{-5}$, 
    close to the real value.
  \end{itemize}
\end{frame}
\begin{frame}
  \frametitle{Exercise}
  Under certain conditions, $6H_2(g) + 2CO_2(g) \rightleftharpoons C_2 H_5 OH(g) + 3H_2O(g)$, the equilibrium 
  percentage conversion of $CO_2$ is as the figure shows. Which one(s) of the following is(are) correct?
  \begin{columns}
    \column{0.5\textwidth}
    \begin{itemize}
      \item[A.] The mole fraction of $CO_2$: $a > b$
      \item[B.] The mole fraction of $C_2H_5OH$: $a < c$
      \item[C.] Equilibrium constant: $K_a > K_c > K_b$
      \item[D.] Reaction rate: $v_a(CO_2) < v_b(CO_2)$  
    \end{itemize}
    \column{0.5\textwidth}
    \includegraphics[width=0.8\linewidth]{assets/exercise_1.jpg}
  \end{columns}
  \pause
  Ans: D.
\end{frame}
\section{Acid-Base}
\begin{frame}
  \frametitle{Definitions}
  Arrhenius: An acid produces $H^{+}$ and a base produces $OH^{-}$ in water. \\
  Bronsted-Lowry: An acid is a proton donor and a base is a proton acceptor. \\
  Lewis: An acid is an electron pair acceptor and a base is an electron pair donor.
  \blankline
  An acid becomes its conjugate base after donating a proton. 
  A base becomes its conjugate acid after accepting a proton.
  \blankline
  \textit{Amphoteric oxides} can react with either an acid or a base. Example: $Al_2O_3$ \\
  \textit{Amphiprotic substances} can either donate or accept a proton. Example: $H_2O$
\end{frame}
\begin{frame}
  \frametitle{Equilibrium Constant for Water}
  \textit{Autoprotolysis} constant of water is
  \begin{equation*}
    K_w = [H^+][OH^-] = 10^{-14}
  \end{equation*}
  $K_w$ increases with temperature because the dissociation of water is endothermic.
  \\[6pt]
  pH is defined as $-\log[H^+]$. pOH is defined as $-\log[OH^-]$. 
  \\[6pt]
  When calculating the pH of a solution of some acid or base, we usually neglect the contribution of the 
  autoprotolysis of water. However, this is not always true. 
  \\[6pt]
  Example: Calculate the pH of $1\times 10^{-7} M$ NaOH solution.
\end{frame}
\begin{frame}
  \frametitle{Weak Acids and Bases}
  For a weak acid $HA$, we define
  \begin{equation*}
    K_a = \frac{[H_3O^+][A^-]}{[HA]}, \quad pK_a = -\log K_a
  \end{equation*}
  Similar for a weak base. 
  For a weak acid $HA$, we can write down the $K_a$ and the $K_b$ for its conjugate base.
  \begin{equation*}
    K_a = \frac{[H_3O^+][A^-]}{[HA]}, \quad K_b = \frac{[HA][OH^-]}{[A^-]}
  \end{equation*}
  Note that
  \begin{equation*}
    K_a K_b = K_w, \quad pK_a + pK_b = 14
  \end{equation*}
  Percentage deprotonated = $\frac{[A^-]}{[HA]_{ini}} \times 100\%$ measures how manually
  acid molecules dissociate.
\end{frame}
\begin{frame}
  \frametitle{Acid Strength}
  There is no general theory, but usually increasing electronegativity 
  leads to the increase of acid strength of $H_nX$. 
  \begin{itemize}
    \item Same Period: Acid strength increases from left to right.
    \item Same Group: Acid strength increases from top to bottom. 
  \end{itemize}
  For oxoacids, in groups with the same number of oxygen atoms, the greater the electronegativity the greater the acidity;
  in a family of the same element, the greater the number of oxygen atoms the greater the acidity. 
  \blankline
  In addition, resonances may also affect the acidity. (usually not required)
\end{frame}
\begin{frame}
  \frametitle{Polyprotic Acids and Bases}
  A polyprotic acid can donate more than one proton. Usually, $K_{a1} >> K_{a2} >> K_{a3}$. 
  \blankline
  To calculate the concentrations of all the solute species in a polyprotic acid solution, we can assume 
  that the concentrations of species with a greater amount are not affected significantly by the species with a smaller amount.
  \\
  Example: For $H_3PO_4$, only the first dissociation affects the concentration of $H_2PO_4^{-}$. 
  \blankline
  Concentrations of species present in a $H_2A$ solution vary with pH:
  \begin{align*}
    \frac{[H_2A]}{c_{ini}} &= \frac{[H_3O^+]^2}{[H_3O^+]^2 + [H_3O^+]K_{a1} + K_{a1}K_{a2}} \\
    \frac{[HA^-]}{c_{ini}} &= \frac{[H_3O^+]K_{a1}}{[H_3O^+]^2 + [H_3O^+]K_{a1} + K_{a1}K_{a2}} \\
    \frac{[A^{2-}]}{c_{ini}} &= \frac{K_{a1}K_{a2}}{[H_3O^+]^2 + [H_3O^+]K_{a1} + K_{a1}K_{a2}}
  \end{align*}
\end{frame}
\begin{frame}
  \frametitle{Balance Equations}
  There are three important balance relation. Take $Na_2CO_3$ as an example.
  \begin{itemize}
    \item Charge-balance relation: $[H^+] + [Na^+] = [OH^-] + [HCO_3^{-}] + 2[CO_3^{2-}]$
    \item Material-balance relation: $[Na^+] = 2([H_2CO_3] + [HCO_3^{-}] + [CO_3^{2-}])$
    \item Proton-balance relation: $[H^+] + [HCO_3^{-}] + 2[H_2CO_3] = [OH^-]$ 
  \end{itemize}
  You can view these relations as another form of ICE table. \\
  Some conclusions:
  \begin{itemize}
    \item $[H_3O^+]^2 - [HCl]_{ini}[H_3O^+] - K_w = 0$ for any strong acid.
    \item $[H_3O^+]^2 - [NaOH]_{ini}[H_3O^+] - K_w = 0$ for any strong base.
    \item $[H_3O^+]^3 + K_a [H_3O^+]^2 - (K_w + K_a[HA]_{ini})[H_3O^+] - K_aK_w = 0$ for weak acids. 
  \end{itemize}
  How to prove them? (Not required)
\end{frame}
\begin{frame}
  \frametitle{Exercise}
  Given that for $H_2S$, $K_{a1} = 1.3\times 10^{-7}$, and $K_{a2} = 7.1\times 10^{-15}$. Calculate the 
  concentration of all solute species in $0.20\ M$ $H_2S(aq)$.
  \pause
  \blankline
  Ans: \\
  $[H_2S] = c = 0.20 M$ \\
  $[H_3O^+] = [HS^-] = \sqrt{cK_a} = 1.6\times 10^{-4}$ M \\
  $[OH^-] = K_w/[H_3O^+] = 6.2\times 10^{-11}$ \\
  $[S^{2-}] = \frac{[HS^-]}{[H_3O^+]}K_{a2} = K_{a2} = 7.1\times 10^{-15}$
\end{frame}
\begin{frame}
  \frametitle{Exercise}
  Estimate the pH of $2.0\times 10^{-4}$ M HCN(aq) given that for HCN, $K_a = 4.9\times 10^{-10}$.
  \pause
  \blankline
  Ans: Plug the numbers into the equation
  \begin{equation*}
    [H_3O^+]^3 + K_a [H_3O^+]^2 - (K_w + K_a[HA]_{ini})[H_3O^+] - K_aK_w = 0
  \end{equation*}
  and get
  \begin{equation*}
    [H_3O^+] = 3.28\times 10^{-7} M, \quad pH = 6.48
  \end{equation*}
\end{frame}
\section{Homework Problems}
\begin{frame}
  \frametitle{C09Q09}
  \includegraphics[width=\linewidth]{assets/C09Q09.png}
\end{frame}
\begin{frame}
  \frametitle{C10Q12}
  \includegraphics[width=\linewidth]{assets/C10Q12.png}
\end{frame}
% insert a reference frame before the 'thank you' frame ----------------------
\begin{frame}
  \frametitle{References}
  
  \begin{thebibliography}{99} % Beamer does not support BibTeX so references must be inserted manually as below
  \bibitem{slides}
  Milias Liu, CHEM2100J slides, Fall 2023.
    
  \bibitem{eiram2013cvssv2}
  Peter Atkins, et. al., \textit{Chemical Principles}, 7th Ed. ISBN: 978-1-4641-8395-9
    
  \end{thebibliography}
\end{frame}

\begin{frame}
  \Huge{\centerline{Thank you!}}
\end{frame}


\end{document}
