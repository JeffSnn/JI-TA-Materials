%----------------------------------------------------------------------------------------
%	TITLE PAGE
%----------------------------------------------------------------------------------------
\documentclass[aspectratio=169]{beamer}
% use 16:9 in new slides!

\usepackage{tikz}
\usepackage{array}
\usepackage{amssymb}

\usepackage[skins]{tcolorbox}

\def\blankline{\\[12pt]}
\def\bpr#1#2{
\begin{tcolorbox}
[boxsep=.15cm,left=.2cm,right=.2cm,oversize,boxrule=0mm,
colback=green!55!blue!20!white!60,
colframe=red!50!yellow!50!white,
colbacktitle=blue!50, 
coltitle=black,enhanced,drop fuzzy shadow,
fonttitle=\bfseries,title=#1]
#2
\end{tcolorbox}
}

\def\bdef#1#2{
\begin{tcolorbox}
[boxsep=.15cm,left=.2cm,right=.2cm,oversize,boxrule=0mm,
colback=white!60,
colframe=red!50!yellow!50!white,
colbacktitle=green!50!yellow!60!gray, 
coltitle=black,enhanced,drop fuzzy shadow,
fonttitle=\bfseries,title=#1]
#2
\end{tcolorbox}
}

\def\bthe#1#2{
\begin{tcolorbox}
[boxsep=.15cm,left=.2cm,right=.2cm,oversize,boxrule=0mm,
colback=white!60,
colframe=red!50!yellow!50!white,
colbacktitle=red!52, 
coltitle=black,enhanced,drop fuzzy shadow,
fonttitle=\bfseries,title=#1]
#2
\end{tcolorbox}
}

\def\tcb#1{
\begin{tcolorbox}
[boxsep=.15cm,left=.2cm,right=.2cm,oversize,boxrule=0mm,
colback=white!60,
colframe=red!50!yellow!50!white,
colbacktitle=red!50!yellow!50!white, coltitle=black,enhanced,drop fuzzy shadow,]
#1
\end{tcolorbox}
}



\definecolor{UMBlue}{RGB}{5,32,103}
\definecolor{UMYellow}{RGB}{255,216,0}
%\usetheme{Madrid}
\setbeamertemplate{headline}{
  \leavevmode%
  \begin{minipage}{0.75\paperwidth}
  \vspace{1.2ex}\hspace{-0.245\paperwidth}
  \resizebox{\paperwidth}{3ex}{
    \tikz{
      \fill [color=UMBlue] (0,0) rectangle (10, 0.13);
      \fill [color=UMYellow] (0,-0.1) rectangle (10, -0.18);
    }
  }
  \begin{beamercolorbox}[wd=\linewidth,ht=2.5ex,dp=1.125ex]{section}
    \insertsubsectionnavigationhorizontal{\linewidth}{}{Slide \insertpagenumber}
  \end{beamercolorbox}
  \end{minipage}
  \begin{minipage}{0.23\paperwidth}
  \hspace{0.2em}
  \includegraphics[width=\linewidth]{assets/logo.png}
  \end{minipage}
}

\setbeamercolor{title}{fg=UMBlue}
\setbeamercolor{frametitle}{fg=UMBlue}
\setbeamercolor{structure}{fg=UMBlue}

\title[Course number]{MATH203 RC Week 3 Friday} 
\author[]{Sun Qizhen}
\institute[UMJI-SJTU]
{
	University of Michigan - Shanghai Jiaotong University
	\\\medskip
	Joint Institute
}
%----------------------------------------------------------------------------------------
%	Highlight the title of the current section
%----------------------------------------------------------------------------------------
\AtBeginSection[]
{
  \begin{frame}
    \frametitle{Table of Contents}
    \tableofcontents[currentsection]
  \end{frame}
}



\begin{document}
% insert title page---------------------------
\maketitle
%insert contents------------------------------
\begin{frame}
  \frametitle{Table of Contents}
  \tableofcontents
\end{frame}
\section{Introduction to My RC}

\begin{frame}
  \frametitle{About Me}
  You may contact me via
  \begin{itemize}
    \item Email: \href{mailto:qz_sun@sjtu.edu.cn}{qz\_sun@sjtu.edu.cn}
    \item Feishu: Search me by my name
    \item WeChat ID: \texttt{Scojesu\_16}
  \end{itemize}
  Feel free to ask any questions about \LaTeX, the course materials, my RCs, or anything else.

\end{frame}
\begin{frame}
  \frametitle{About My RC}
  I usually upload my slides several days before the RC. My RCs may differ from the ones held by the other
  two TAs in the following aspects. 
  \begin{itemize}
    \item Contain less content about basic concepts. I assume you have attended the lectures, and only provide a basic review of the 
    concepts unless they are really hard to comprehend.
    \item Include more exercise. I have provided a set of problems that may be used in my RC in the file section in canvas.
    You are encouraged to try them yourselves before the RC.
    \item Talk about stuff that beyond the course scope, or not required in the exams sometimes. However, they will be marked by a star 
    such that you can skip them if you are not interested.
  \end{itemize}
\end{frame}
\section{Logic}
\begin{frame}
  \frametitle{Things You Should Know}
  \begin{itemize}
    \item Cartesian Product: $A\times B$
    \item Power Set: $\mathcal{P}(A)$ or $2^A$
    \item All basic logic operators: $\land, \lor, \lnot, \oplus, \rightarrow, \Rightarrow$
    \item Proof technique: Proof by Contradiction
    \item Logical Quantifiers
    \item Truth Trees (aka. Semantic Tableau) for First-Order Logic
    \item Natural Deduction
  \end{itemize}
\end{frame}
\begin{frame}
  \frametitle{What You Need NOT Know}
  You may be curious about what will be tested in the exam.
  Usually, there will be one problem concerning the truth tree and another one concerning the natural deduction in the midterm. 
  In addition, 
  \begin{itemize}
    \item Anything about Lean will not be covered in the exam
    \item Graph theory was removed from this course
    \item Rules for natural deduction will be provided in the exam
  \end{itemize}
  If you are not sure whether some theorem or conclusion will be tested, feel free to ask (better ask on piazza).
\end{frame}
\begin{frame}
  \frametitle{Exercise}
  I believe the logic part is not hard for you. So let us directly jump to the exercise part.
  \blankline
  \textbf{Ex. 1} Prove the followings using truth trees.
  \begin{itemize}
      \item[(a)] $\vdash \lnot \forall x \exists y \left(\lnot P(x) \land P(y)\right)$
      \item[(b)] $\vdash \exists x \forall y \left(P(x) \lor \lnot P(y)\right)$
  \end{itemize}
  \vspace*{20pt}
  You may try on \url{ https://www.umsu.de/trees/} yourself for the solution.
\end{frame}
\begin{frame}
  \frametitle{Solution}
\end{frame}
\begin{frame}
  \frametitle{Exercise}
  \textbf{Ex. 2} Use a natural deduction tree to prove 
  \begin{equation*}
      \left(\lnot P \lor Q \right) \vdash \left(P \rightarrow Q\right)
  \end{equation*}
  and the other way around, i.e. 
  \begin{equation*}
      \left(P \rightarrow Q\right) \vdash \left(\lnot P \lor Q \right)
  \end{equation*}
\end{frame}
\begin{frame}
  \frametitle{Solution}
\end{frame}
\section{Mathematical Induction}
\begin{frame}
  \frametitle{Basic Induction}
  Induction on $\mathbb{N}$. 
  \begin{itemize}
    \item \textbf{Base Case}: $P(0)$ is true 
    \item \textbf{Inductive Case}: $\forall k\in \mathbb{N}, P(k)\rightarrow P(k + 1)$ 
    \item \textbf{Conclusion}: $P(n)$ is true for all $n\in \mathbb{N}$ 
  \end{itemize}
  Strong Induction on $\mathbb{N}$
  \begin{itemize}
    \item \textbf{Base Case}: $P(0)$ is true 
    \item \textbf{Inductive Case}: $\forall k\in \mathbb{N}, P(0) \land P(1) \land \cdots \land P(k) \rightarrow P(k + 1)$ 
    \item \textbf{Conclusion}: $P(n)$ is true for all $n\in \mathbb{N}$ 
  \end{itemize}
\end{frame}
\begin{frame}
  \frametitle{Monoid}
  A \textit{monoid} is a triple $\left(M, *, e\right)$ where $M$ is a set, $e\in M$ and $*: M\times M\to M$ such that
  for all $x, y, z\in M$
  \begin{itemize}
    \item Identity Element: $x * e = e * x = x$
    \item Associativity: $\left(x * y\right) * z = x * \left(y * z\right)$
  \end{itemize}
  The identity element is unique for a monoid.
  \\[12pt]
  Our course will cover basics in group theory later. Now you only need to know the definitions and some basic
  properties of these abstract structures.
\end{frame}
\begin{frame}
  \frametitle{Recursively Defined Structures}
  A natural number is either
  \begin{itemize}
    \item zero
    \item the successor of a natural number $n$
  \end{itemize}
  \vspace*{12pt}
  The set of all natural numbers is defined as
  \begin{itemize}
    \item $0\in\mathbb{N}$
    \item If $n\in\mathbb{N}$, then $n^+ \in \mathbb{N}$
  \end{itemize}
  $\mathbb{N}$ is defined as the \textit{smallest set} satisfy the two rules. By default, we consider
  the set being defined the smallest set containing these elements.
\end{frame}
\begin{frame}
  \frametitle{Structural Induction}
  The set $\Sigma^*$ of \textit{strings} over the alphabet $\Sigma$ is defined recursively by
  \begin{itemize}
    \item The empty string $\varepsilon \in \Sigma^*$
    \item If $a\in\Sigma$ and $x\in\Sigma^*$, $ax:=\left(a, x\right)\in\Sigma^*$
  \end{itemize}
  $ax := \left(a, x\right)$ is very important in proofs. \\[12pt]
  I do not include exercises about structural induction on strings because they all look similar and are not interesting. However, 
  please do review the examples in slides and your homework, because usually there will be more than 1 problems
  concerning this part in your midterm. 
\end{frame}
\begin{frame}
\frametitle{Structural Induction on Strings}
  I still provide some conceptual exercises as follows.
  \begin{itemize}
    \item How to define the function that reverses a string?
    \item How to prove a string reversed twice is the same as the original string using your definitions?
    \item How to define a \textit{palindrome} string?
    \item How to prove the properties of palindrome strings (e.g. at most one character appears odd times) using your definitions? 
    \item How to define a substring?
  \end{itemize}
\end{frame}
\begin{frame}
  \frametitle{Well-Ordering Principle}
  WOP is stated as follows.
  \begin{quotation}
    Every non-empty subset of $\mathbb{N}$ has a least element.
  \end{quotation}
  It is useful sometimes when you are trying to proving the existence of something. Though I think it is 
  not necessary for you to use it in most cases, you should still at least know it. 
\end{frame}
\begin{frame}
  \frametitle{Exercise}
  Let us come to our final exercise in this RC. I believe this problem interests you. \\
  \textbf{Ex. 5} (Gallier 1.45) A set $A$ is said to be \textit{transitive} if for all $a\in A$, 
  \begin{equation*}
      a\in A \Rightarrow a\subseteq A
  \end{equation*}
  \begin{itemize}
      \item[(a)] Show that a set $A$ is transitive if and only if
      \begin{equation*}
          \bigcup A \subseteq A
      \end{equation*}
      \item[(b)](von Neumann successor) We define
      \begin{equation*}
          A^+ = A\cup \{A\}
      \end{equation*}
      Prove that if $A$ is a transitive set, then
      \begin{equation*}
          \bigcup \left(A^+\right) = A
      \end{equation*}
  \end{itemize}
\end{frame}
\begin{frame}
  \begin{itemize}
      \item[(c)] Recall that the set of natural numbers can be recursively defined as
      \begin{align*}
          &\mbox{(i) }\ 0 := \varnothing \in \mathbb{N}, \\
          &\mbox{(ii) For any } n \in \mathbb{N}, n^+ := n\cup \{n\} \in \mathbb{N}.
      \end{align*}
      Prove that every natural number is a transitive set.
      \item[(d)] Show that for two natural numbers $m$ and $n$, if $m^+ = n^+$, then $m = n$.
      \item[(e)] Show that $\mathbb{N}$, the set of all natural numbers, is a transitive set. 
  \end{itemize}
\end{frame}
\begin{frame}
  \frametitle{Solution}
\end{frame}
\section{Homework}
\begin{frame}
  \frametitle{Homework}
  This part will be updated after I have finished grading the homework.
\end{frame}
\section{Appendix: Solution to Problem Set}
\begin{frame}
  \frametitle{Ex. 1}
  Try on \url{ https://www.umsu.de/trees/} yourself
\end{frame}
\begin{frame}
  \frametitle{Ex. 2}
  See slides pp. 82-84. 
\end{frame}
\begin{frame}
  \frametitle{Ex. 3}
  $P(k)$: Both sides are equivalent for $n=k$.
  Inductive step: We have
  \begin{align*}
    a_1 \land a_2 \cdots \land a_k \land a_{k+1} \rightarrow b&= a_1 \land \left(a_2 \cdots \land a_k \land a_{k+1}\right) \rightarrow b \\
    &=a_1\rightarrow \left(\left(a_2 \cdots \land a_k \land a_{k+1}\right) \rightarrow b\right) \\
    &=a_1 \rightarrow \left(a_2 \rightarrow \cdots \rightarrow \left(a_{k+1} \rightarrow b\right)\right)
  \end{align*}
  where we applied our IH in the last step.
\end{frame}
\begin{frame}
  \frametitle{Ex. 4}
  You need to prove the following properties of xor.
  \begin{itemize}
    \item Commutativity: $a\oplus b = b\oplus a$. 
    \item Associativity: $\left(a\oplus b\right)\oplus c = a\oplus \left(b\oplus c\right)$. 
    \item Identity Element: $a\oplus 0 = a$.
    \item Inverse Element: $a\oplus a = 0$.
  \end{itemize}
  You may view xor as the function 
  \begin{equation*}
    \oplus: \mathbb{B}\times \mathbb{B} \to \mathbb{B}, \ \left(x, y\right)\mapsto \left(x + y\right)\mod 2
  \end{equation*}
  and these properties are trivially true.
\end{frame}
\begin{frame}
  \frametitle{Ex. 5}
  \begin{itemize}
    \item[(a)] ($\Rightarrow$) Suppose $x\in\bigcup A$, there exists $a\in A$ such that $x\in a$. Since $A$ is transitive, $a\in A$ implies
    $a\subseteq A$. $x\in a \land a\subseteq A \Rightarrow x\in A$. Hence $\bigcup A \subseteq A$.\\
    ($\Leftarrow$) Suppose $a\in A$. For any $x\in a$, $x\in \bigcup A$. Since $\bigcup A\subseteq A$, $x\in A$. Then $a\subseteq A$.
    \item[(b)] Suppose $A$ is transitive, by (a) we have $\bigcup A \subseteq A$.
    \begin{equation*}
      \bigcup \left(A^+\right) = \bigcup \left(A\cup \{A\}\right) = \left(\bigcup A\right) \cup A = A
    \end{equation*}
    \item[(c)] Proof by Induction. \\
    \textbf{Base Case}: $\varnothing $ is transitive because $\bigcup \varnothing = \varnothing \subseteq \varnothing$ is vacuously true. \\
    \textbf{Inductive Case}: Say $n = m^+ = m \cup \{m\}$. By IH, $m$ is transitive. We want to show $n$ is also transitive. \\
    For any $x\in n = m \cup \{m\}$, either
    \begin{itemize}
      \item $x\in m$. By transitivity of $m$, $x\subseteq m$. Hence $x\subseteq m \cup \{m\}$.
      \item $x = m$. Then trivially $m\subseteq m \cup \{m\}$.
    \end{itemize}
    We can now conclude that $x\subseteq n$, and it follows that $n$ is also transitive.
  \end{itemize}
\end{frame}
\begin{frame}
  \frametitle{Ex. 5}
  \begin{itemize}
    \item[(d)] Suppose $m\cup \{m\} = n\cup \{n\}$. Suppose for contradiction that $m\neq n$. Then, $m\in n$ and $n\in m$. In (c) we have 
    showed that all natural numbers are transitive, so $m\subseteq n$ and $n\subseteq m$. Hence $m = n$.
    \item[(e)] It is equivalent to show for all $x\in \mathbb{N}$, $x\subseteq \mathbb{N}$. Again, we prove it by induction. \\
    \textbf{Base Case}: $\varnothing \subseteq \mathbb{N}$ is vacuously true. \\
    \textbf{Inductive Case}: Say $n = m^+ = m\cup \{m\}$. By IH, $m\subseteq \mathbb{N}$. We want to show $n\subseteq \mathbb{N}$. \\
    For any $x\in n = m\cup \{m\}$, either
    \begin{itemize}
      \item $x\in m$. $m\subseteq \mathbb{N}$ implies $x\in \mathbb{N}$. 
      \item $x = m$. $x\in\mathbb{N}$ since $m\in \mathbb{N}$.
    \end{itemize}
    We can now conclude that $x\in \mathbb{N}$, and it follows that $n\subseteq \mathbb{N}$.
  \end{itemize}
\end{frame}
\begin{frame}
  \frametitle{Ex. 6}
  We claim that for $n$ layers, our score is $\frac{n\left(n-1\right)}{2}$ despite the strategy. Proof by strong induction. \\
  \textbf{Base Case}: For $n=1$, the score is 0. \\
  \textbf{Inductive Case}: By IH, the score for $x$ layers ($x\leq n$) is $\frac{x\left(x-1\right)}{2}$. 
  For $n + 1$ layers, we may split them into two sub-stacks containing $k$ and $n + 1 - k$ layers respectively. Then, 
  our final total score will be
  \begin{equation*}
    \frac{k\left(k-1\right)}{2} + \frac{\left(n+1-k\right)\left(n-k\right)}{2} + k\left(n+1-k\right) = \frac{n\left(n+1\right)}{2}
  \end{equation*}
  So IH holds for $n + 1$ layers also. We can conclude that it holds for all $n\in\mathbb{N}$. 
\end{frame}
% insert a reference frame before the 'thank you' frame ----------------------
\begin{frame}
  \frametitle{References}
  
  \begin{thebibliography}{99} % Beamer does not support BibTeX so references must be inserted manually as below
    \bibitem[1]{gallier} Jean Gallier. Discrete Mathematics. Dordrecht; Heidelberg, Springer, 2011. ISBN: 978-1-4419-8046-5 
    \bibitem[2]{wanling} Wanling Qu, Suyun Geng, Liang Zhang. Discrete Mathematics. Tsinghua University, 2013. ISBN: 978-7-302-33989-2
    \bibitem[3]{mitCourseNotes} Thomson Leighton, Albert R Meyer. Mathematics for Computer Science Course Notes. MIT OpenCourseWare, 2010.
    \bibitem[4]{slides} Prof. Cai. MATH203 Slides.
  \end{thebibliography}
\end{frame}

\begin{frame}
  \Huge{\centerline{Thank you!}}
\end{frame}


\end{document}
