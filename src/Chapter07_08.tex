%----------------------------------------------------------------------------------------
%	TITLE PAGE
%----------------------------------------------------------------------------------------
\documentclass[aspectratio=169]{beamer}
% use 16:9 in new slides!

\usepackage{tikz}
\usepackage{array}

\usepackage[skins]{tcolorbox}

\def\blankline{\\[6pt]}
\def\bpr#1#2{
\begin{tcolorbox}
[boxsep=.15cm,left=.2cm,right=.2cm,oversize,boxrule=0mm,
colback=green!55!blue!20!white!60,
colframe=red!50!yellow!50!white,
colbacktitle=blue!50, 
coltitle=black,enhanced,drop fuzzy shadow,
fonttitle=\bfseries,title=#1]
#2
\end{tcolorbox}
}

\def\bdef#1#2{
\begin{tcolorbox}
[boxsep=.15cm,left=.2cm,right=.2cm,oversize,boxrule=0mm,
colback=white!60,
colframe=red!50!yellow!50!white,
colbacktitle=green!50!yellow!60!gray, 
coltitle=black,enhanced,drop fuzzy shadow,
fonttitle=\bfseries,title=#1]
#2
\end{tcolorbox}
}

\def\bthe#1#2{
\begin{tcolorbox}
[boxsep=.15cm,left=.2cm,right=.2cm,oversize,boxrule=0mm,
colback=white!60,
colframe=red!50!yellow!50!white,
colbacktitle=red!52, 
coltitle=black,enhanced,drop fuzzy shadow,
fonttitle=\bfseries,title=#1]
#2
\end{tcolorbox}
}

\def\tcb#1{
\begin{tcolorbox}
[boxsep=.15cm,left=.2cm,right=.2cm,oversize,boxrule=0mm,
colback=white!60,
colframe=red!50!yellow!50!white,
colbacktitle=red!50!yellow!50!white, coltitle=black,enhanced,drop fuzzy shadow,]
#1
\end{tcolorbox}
}



\definecolor{UMBlue}{RGB}{5,32,103}
\definecolor{UMYellow}{RGB}{255,216,0}
%\usetheme{Madrid}
\setbeamertemplate{headline}{
  \leavevmode%
  \begin{minipage}{0.75\paperwidth}
  \vspace{1.2ex}\hspace{-0.245\paperwidth}
  \resizebox{\paperwidth}{3ex}{
    \tikz{
      \fill [color=UMBlue] (0,0) rectangle (10, 0.13);
      \fill [color=UMYellow] (0,-0.1) rectangle (10, -0.18);
    }
  }
  \begin{beamercolorbox}[wd=\linewidth,ht=2.5ex,dp=1.125ex]{section}
    \insertsubsectionnavigationhorizontal{\linewidth}{}{Slide \insertpagenumber}
  \end{beamercolorbox}
  \end{minipage}
  \begin{minipage}{0.23\paperwidth}
  \hspace{0.2em}
  \includegraphics[width=\linewidth]{assets/logo.png}
  \end{minipage}
}

\setbeamercolor{title}{fg=UMBlue}
\setbeamercolor{frametitle}{fg=UMBlue}
\setbeamercolor{structure}{fg=UMBlue}

\title[Course number]{VC210 Thermodynamics} 
\author[]{Sun Qizhen}
\institute[UMJI-SJTU]
{
	University of Michigan - Shanghai Jiaotong University
	\\\medskip
	Joint Institute
}
%----------------------------------------------------------------------------------------
%	Highlight the title of the current section
%----------------------------------------------------------------------------------------
\AtBeginSection[]
{
  \begin{frame}
    \frametitle{Table of Contents}
    \tableofcontents[currentsection]
  \end{frame}
}



\begin{document}
% insert title page---------------------------
\maketitle
%insert contents------------------------------
\begin{frame}
  \frametitle{Table of Contents}
  \tableofcontents
\end{frame}


\section{Work and Heat}
% insert a sample frame without animation--------------------------------
  \begin{frame}
  \frametitle{System, Work, and Reversible Process}
  Three kinds of systems:
  \begin{itemize}
    \item open: matter and energy exchanged with surroundings
    \item closed: energy only
    \item isolated: neither can be exchanged
  \end{itemize}
  Two kinds of work:
  \begin{itemize}
    \item expansion work: work involves a \textbf{change of volume}
    \item non-expansion work: otherwise
  \end{itemize}
  Free expansion: expansion with no external pressure.
  \blankline
  All real-world thermodynamics processes are \textbf{irreversible}. Changes brought by a \textit{reversible} 
  process can be cancelled by some process, and there is no overall changes to the surroundings. You may refer to \href{https://zhuanlan.zhihu.com/p/33133532}{this link}(click on it) for further information.

  \end{frame}
  \begin{frame}
    \frametitle{First Law of Thermodynamics}
    $$\Delta U = Q + W$$
    where $Q$ is the heat \textbf{absorbed} by the system, $W$ is the work done \textbf{to} the system. You can also write it as $\Delta U = Q - W$ where $W$ is the work done \textbf{by} the system.
    \blankline
    Corollary: The internal energy of an isolated system is constant.
    \blankline
    For expansion work, 
    $$dW = -P_{ex}dV$$
    More about internal energy: $U$ consists of molecular kinetic and potential energy. An estimate (vibrational energy neglected) 
    value of molar internal energy is given by $\frac{5RT}{2}$ for linear molecules and $\frac{6RT}{2}$ for nonlinear molecules. 
  \end{frame}
  \begin{frame}
    \frametitle{Exercise}
    Water expands when it freezes. How much work is done \textit{to} 100g water when it freezes at $0^{\circ}C$ and pushes back the metal wall
    of a pipe that exerts an opposing pressure of 1070 atm? The densities of water and ice at $0^{\circ}C$ are $1.00\ g\cdot cm^{-3}$ 
    and $0.92\ g\cdot cm^{-3}$ respectively. \pause
    \blankline
    Answer: -943J

    We simply calculate the volume difference and multiply it with the pressure and get our answer.
    Pay attention to the negative sign here.
  \end{frame}
  \begin{frame}
    \frametitle{Heat Capacity}
    $$C = \frac{Q}{\Delta T}$$
    The specific heat capacity and molar heat capacity are respectively
    $$C_{s} = \frac{C}{m}, \qquad C_m = \frac{C}{n} $$
  \end{frame}
\section{Enthalpy}
  \begin{frame}
    \frametitle{Enthalpy}
    \textit{Enthalpy} is defined as follows. It is also a state function. Actually, it has no obvious physical meaning at all.
    $$H := U + pV$$
    Under constant pressure
    $$\Delta H = Q$$
    Under constant volume
    $$\Delta U = Q$$
    In chemical reactions, $\Delta H$ stands for the heat change during the reaction. $\Delta H < 0$ means the reaction is \textit{exothermic}; $\Delta H > 0$ means the reaction is \textit{endothermic}. Recall that $Q$ stands for the heat \textbf{absorbed}.
  \end{frame}
  \begin{frame}
    \frametitle{More about Heat Capacity}
    Note that
    \begin{align*}
      C_{V, m} = \frac{Q}{\Delta T} &= \frac{\Delta U}{\Delta T}, &\text{const volume} \\
      C_{p, m} = \frac{Q}{\Delta T} &= \frac{\Delta H}{\Delta T} = \frac{\Delta U + pV}{\Delta T} = C_V + nR, &\text{const pressure}
    \end{align*}
    Copy the following table to your cheating paper!
    \begin{table}[H]
      \centering
      \begin{tabular}{c | c | c | c}
          & Monoatomic & Linear & Nonlinear \\[4pt]
        \hline
        $C_{V, m}$ & $\frac{3}{2}R$ & $\frac{5}{2}$R & $3R$ \\[3pt]
        $C_{p, m}$ & $\frac{5}{2}R$ & $\frac{7}{2}$R & $4R$
      \end{tabular}
    \end{table}
  \end{frame}
  \begin{frame}
    \frametitle{Enthalpy in Chemical Changes}
    $\Delta U$ and $\Delta H$ is different when gas is formed. 
    $$\Delta H = \Delta U + \Delta n_{gas}RT $$
    \textit{Hess's Law}: For multiple reactions, we can add up the enthalpies when combining up the reactions.
    \blankline
    \textit{Standard Enthalpy of Formation}: Denoted by $\Delta H_{f}^{\circ}$. Change in enthalpy for forming \textbf{one mole} molecule in standard state ($25^{\circ}C$, 1 bar). 
    We usually choose the elements in their most stable form. One exception is Phosphorous.
    \blankline
    The standard reaction enthalpy can then be calculated as
    $$\Delta H^{\circ} = \sum n \Delta H_{f}^{\circ}(\text{products}) - \sum n \Delta H_{f}^{\circ}(\text{reactants})$$
  \end{frame}
  \begin{frame}
    \frametitle{Enthalpy in Chemical Changes}
    Reaction enthalpy with temperature: 
    $$\Delta H_{T_2}^{\circ} - \Delta H_{T_1}^{\circ} = (T_2 - T_1)\left(\sum n C_{p, m}(\text{products}) - \sum n C_{p, m}(\text{reactants}) \right)$$
    Born-Haber Cycle: An application of Hess's Law.
    \blankline
    Heating Curve: A temperature-heat curve. When the state of the sample is changing, 
    temperature is unchanged. The slope stands for the heat capacity at that stage, i.e.
    $$\frac{dT}{dq} = C$$
  \end{frame}
  \begin{frame}
    \frametitle{Exercise}
    Calculate the final temperature and the enthalpy change when 500.J of heat is transferred to 0.900 mol $O_2$ at 298K and 1.00 atm at
    \begin{itemize}
      \item (a) constant pressure
      \item (b) constant volume
    \end{itemize}
    You should assume $O_2$ behaves as an ideal gas and there is no vibrational contribution to the molar heat capacity.
    \pause
    \blankline
    Answer: (a) 317K, 500.J; (b) 325K, 700.J 

    Pay attention to the dots after $500$ and $700$, they indicate that the two numbers have 3 significant digits.
  \end{frame}
\section{Entropy}
% insert a sample frame with animation 2 -----------------------------------
  \begin{frame}
    \frametitle{Calculations about entropy}
    Entropy change in a system (under const pressure, $Q_{rev} = \Delta H$):
    \begin{equation*}
      \Delta S = \frac{Q_{rev}}{T}
    \end{equation*}
    Corollaries:
    \begin{align*}
      \Delta S &= C\ln\left(\frac{T_2}{T_1}\right), \ \text{$C$ is the heat capacity} \\
      \Delta S &= nR\ln\left(\frac{V_2}{V_1}\right) \\
      \Delta S &= nR\ln\left(\frac{P_1}{P_2}\right)
    \end{align*}
    We can calculate the entropy change by separating the process into several steps, each of which changes only one 
    related quantity ($T$, $V$, or $p$).
  \end{frame}
  \begin{frame}
    \frametitle{Boltzmann's Formula}
    Statistical entropy ($W$ is the number of microstates):
    \begin{equation*}
      S = k \ln W
    \end{equation*}
    For $n$ molecules, if each of the molecule has $t$ kinds of orientation, 
    \begin{equation*}
      S = k \ln t^n = nk \ln t
    \end{equation*}
    A relation between $k$ and the constants you are familiar with:
    \begin{equation*}
      k = R/N_A
    \end{equation*}
    where $R$ is the ideal gas constant, and $N_A$ is the Avogadro constant.
  \end{frame}
  \begin{frame}
    \frametitle{Standard Molar Entropies}
    Our motivation is to integrate the following formula with the condition that $S=0$ at $T=0$ for perfect crystals.
    \begin{equation*}
      dS = \frac{CdT}{T}
    \end{equation*}
    Note that $C$ is the heat capacity, and it usually changes with temperature under low temperatures.
    \blankline
    Complex and heavy molecules have higher standard molar entropy. Also, 
    \begin{equation*}
      S_{m, gas}^{\circ} > S_{m, liquid}^{\circ} > S_{m, solid}^{\circ}
    \end{equation*}
    We can calculate the standard entropy change of a reaction by
    \begin{equation*}
      \Delta S_{system}^{\circ} = \sum n S_{m, products}^{\circ} - \sum n S_{m, reactants}^{\circ}
    \end{equation*}
  \end{frame}
  \begin{frame}
    \frametitle{Total Entropy}
    The total entropy of a system and its surroundings is
    \begin{equation*}
      \Delta S_{total} = \Delta S_{system} + \Delta S_{surroundings} = \Delta S_{system} - \frac{\Delta H_{system}}{T}
    \end{equation*}
    Note that the second equality requires constant pressure. Also, the surrounding entropy is not a state function.
    \blankline
    A positive value for total entropy indicates that the process is spontaneous. A negative value indicates the reverse process is spontaneous.
    A zero value indicates neither of them is spontaneous.
    \blankline
    In an isolated system, 
    \begin{equation*}
      \Delta S = \frac{Q_{rev}}{T}\geq \frac{Q}{T}\geq 0
    \end{equation*}
  \end{frame}
  \begin{frame}
    \frametitle{Gibbs Free Energy}
    Gibbs free energy is defined as
    \begin{equation*}
      G = H - TS
    \end{equation*}
    It is a state function, and it can tell us whether a process is spontaneous or not.
    \begin{equation*}
      \Delta G = \Delta H - T\Delta S = T(\frac{\Delta H}{T} - \Delta S) = -T(\Delta S_{sys} + \Delta S_{surroundings}) = -T\Delta S_{total}
    \end{equation*}
    $\Delta G_f^{\circ}$ is the standard Gibbs free energy of formation. A positive $\Delta G_{f}^{\circ}$ indicates that elements tend to form the 
    compound. A negative value indicates the compound is thermodynamically unstable. However, the decomposition may be slow, which is called \textit{inert} or \textit{non-labile}.
    \blankline
    With the formula of Gibbs free energy, we can see that a non-spontaneous reaction may become a spontaneous one under certain temperature.

  \end{frame}
  \begin{frame}
    \frametitle{Exercise}
    The temperature of 1.00 mol He(g) is increased from $25.^{\circ}C$ to $300.^{\circ}C$ at constant volume. What is the 
    change in the entropy of the helium?
    
    You should assume ideal behavior and that there is no vibrational contribution to the molar heat capacity.
    \pause
    \blankline
    Answer: 8.15 J/K
    $$\Delta S = nC_{V, m, monoatomic}\ln(\frac{T_2}{T_1}) $$
  \end{frame}
  \begin{frame}
    \frametitle{Exercise}
    1.00 mol $O_2$(g) was compressed suddenly (and irreversibly) from 5.00 L to 1.00 L by 
    driving in a piston, and in the process its temperature was increased from $20.0^{\circ}C$ to $25.2^{\circ}C$. 
    What is the change in entropy of the gas?

    Assume ideal behavior and that there is no vibrational contribution to the molar heat capacity.
    \pause
    \blankline
    Answer: -13.0 J/K

    We can separate the process into two steps.
    \begin{align*}
      \Delta S  &= \Delta S_1 + \Delta S_2 \\
                &= nR\ln(\frac{V_2}{V_1}) + nC_{V, m, diatomic}\ln(\frac{T_2}{T_1}) \\
                &= -13.38 J/K + 0.36 J/K = -13.0 J/K 
    \end{align*}
  \end{frame}
  \begin{frame}
    \frametitle{Exercise}
    Calculate the entropy of vaporization of 1 mol acetone at 296 K with an external pressure of 1 bar. The molar heat capacity of liquid
    acetone is 127 J/(K mol), its boiling point is 329.4 K, and its enthalpy of vaporization is 29.1 kJ/mol.

    Assume ideal behavior and that there is no vibrational contribution to the molar heat capacity.
    \pause
    \blankline
    Answer: 98.4 J/K
    \begin{align*}
      \Delta S  &= \Delta S(heat\ up) + \Delta S(vaporize) + \Delta S(cool\ down) \\
                &= C_{liquid} \ln(\frac{T_2}{T_1}) + \frac{\Delta H_{vap}}{T_{boil}} + C_{gas} \ln(\frac{T_1}{T_2}) \\  
                &= 13.58 J/K + 88.34 J/K - 3.56 J/K = 98.4 J/K
    \end{align*}
    where $C_{gas} = 4R$.
  \end{frame}
  \section{Homework Problems}
  \begin{frame}
    \frametitle{Midterm Q2}
    2-Methyl-1, 3, 5-trinitrobenzene ($C_7H_5N_3O_6$), informally known as TNT is shown on 
    the right. It is a well-known explosive that will release $N_2$, $H_2$, and 
    CO (as well as C in the form of soot) upon ignition.
    \blankline
    (a) What is the reaction equation of the explosion of TNT?
    
    (b) What volume of gas is released upon the explosion of an 
  M107 artillery shell, filled with 6.83 kg of TNT at 25 °C 
  and 1 atm?
  \end{frame}
  \begin{frame}
    \frametitle{Chapter 07 Q18}
    \includegraphics[scale = 0.9]{assets/C07Q18.png}
  \end{frame}
  \begin{frame}
    \frametitle{Chapter 08 Q5}
    \includegraphics[scale = 0.9]{assets/C08Q5.png}
  \end{frame}

% insert a reference frame before the 'thank you' frame ----------------------
\begin{frame}
  \frametitle{References}
  
  \begin{thebibliography}{99} % Beamer does not support BibTeX so references must be inserted manually as below
  \bibitem{slides}
  Milias Liu, CHEM2100J slides, Fall 2023.
    
  \bibitem{eiram2013cvssv2}
  Peter Atkins, et. al., \textit{Chemical Principles}, 7th Ed. ISBN: 978-1-4641-8395-9

  \bibitem{RCslides}
  Jason Chow, CHEM2100J slides, Fall 2022.
    
  \end{thebibliography}
  \end{frame}

\begin{frame}
  \Huge{\centerline{Thank you!}}
\end{frame}


\end{document}
