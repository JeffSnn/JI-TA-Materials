%----------------------------------------------------------------------------------------
%	TITLE PAGE
%----------------------------------------------------------------------------------------
\documentclass[aspectratio=169]{beamer}
% use 16:9 in new slides!

\usepackage{tikz}
\usepackage{array}

\usepackage[skins]{tcolorbox}

\def\blankline{\\[12pt]}
\def\bpr#1#2{
\begin{tcolorbox}
[boxsep=.15cm,left=.2cm,right=.2cm,oversize,boxrule=0mm,
colback=green!55!blue!20!white!60,
colframe=red!50!yellow!50!white,
colbacktitle=blue!50, 
coltitle=black,enhanced,drop fuzzy shadow,
fonttitle=\bfseries,title=#1]
#2
\end{tcolorbox}
}

\def\bdef#1#2{
\begin{tcolorbox}
[boxsep=.15cm,left=.2cm,right=.2cm,oversize,boxrule=0mm,
colback=white!60,
colframe=red!50!yellow!50!white,
colbacktitle=green!50!yellow!60!gray, 
coltitle=black,enhanced,drop fuzzy shadow,
fonttitle=\bfseries,title=#1]
#2
\end{tcolorbox}
}

\def\bthe#1#2{
\begin{tcolorbox}
[boxsep=.15cm,left=.2cm,right=.2cm,oversize,boxrule=0mm,
colback=white!60,
colframe=red!50!yellow!50!white,
colbacktitle=red!52, 
coltitle=black,enhanced,drop fuzzy shadow,
fonttitle=\bfseries,title=#1]
#2
\end{tcolorbox}
}

\def\tcb#1{
\begin{tcolorbox}
[boxsep=.15cm,left=.2cm,right=.2cm,oversize,boxrule=0mm,
colback=white!60,
colframe=red!50!yellow!50!white,
colbacktitle=red!50!yellow!50!white, coltitle=black,enhanced,drop fuzzy shadow,]
#1
\end{tcolorbox}
}



\definecolor{UMBlue}{RGB}{5,32,103}
\definecolor{UMYellow}{RGB}{255,216,0}
%\usetheme{Madrid}
\setbeamertemplate{headline}{
  \leavevmode%
  \begin{minipage}{0.75\paperwidth}
  \vspace{1.2ex}\hspace{-0.245\paperwidth}
  \resizebox{\paperwidth}{3ex}{
    \tikz{
      \fill [color=UMBlue] (0,0) rectangle (10, 0.13);
      \fill [color=UMYellow] (0,-0.1) rectangle (10, -0.18);
    }
  }
  \begin{beamercolorbox}[wd=\linewidth,ht=2.5ex,dp=1.125ex]{section}
    \insertsubsectionnavigationhorizontal{\linewidth}{}{Slide \insertpagenumber}
  \end{beamercolorbox}
  \end{minipage}
  \begin{minipage}{0.23\paperwidth}
  \hspace{0.2em}
  \includegraphics[width=\linewidth]{assets/logo.png}
  \end{minipage}
}

\setbeamercolor{title}{fg=UMBlue}
\setbeamercolor{frametitle}{fg=UMBlue}
\setbeamercolor{structure}{fg=UMBlue}

\title[Course number]{VC210 Aqueous Solution Equilibrium} 
\author[]{Sun Qizhen}
\institute[UMJI-SJTU]
{
	University of Michigan - Shanghai Jiaotong University
	\\\medskip
	Joint Institute
}
%----------------------------------------------------------------------------------------
%	Highlight the title of the current section
%----------------------------------------------------------------------------------------
\AtBeginSection[]
{
  \begin{frame}
    \frametitle{Table of Contents}
    \tableofcontents[currentsection]
  \end{frame}
}



\begin{document}
% insert title page---------------------------
\maketitle
%insert contents------------------------------
\begin{frame}
  \frametitle{Table of Contents}
  \tableofcontents
\end{frame}


\section{Buffer}
\begin{frame}
  \frametitle{Salts of Polyprotic Acids}
  A conclusion about the pH of solutions of salts of polyprotic acids has been mentioned in the lecture.
  \begin{equation*}
    pH = \frac{1}{2}(pK_{a1} + pK_{a2})
  \end{equation*}
  How to prove it? (Not required) \\
  \pause
  Proof: Take NaHS as an example. The process works for any polyprotic acids. 
  \begin{equation*}
    \begin{aligned}
      HS^-(aq) + H_2O(l) &\rightleftharpoons S^{2-} + H_3O^+\\
      HS^-(aq) + H_2O(l) &\rightleftharpoons H_2S + OH^- \\
      \frac{[H^+][S^{2-}]}{[HS^-]} & = K_{a2}, \quad [S^{2-}] = \frac{K_{a2}[HS^-]}{[H^+]} \\
      \frac{[H_2S][OH^-]}{[HS^-]} & = \frac{K_w}{K_{a1}}, \quad [H_2S] = \frac{[HS^-][H^+]}{K_{a1}} \\ 
    \end{aligned}
  \end{equation*}
\end{frame}
\begin{frame}
  \frametitle{Proof}
  From charge-balance relation, 
  \begin{equation*}
    [Na^+] + [H^+] = [OH^-] + 2[S^{2-}] + [HS^-] 
  \end{equation*}
  From material-balance relation, 
  \begin{equation*}
    c := [Na^+] = [H_2S] + [S^{2-}] + [HS^-]
  \end{equation*}
  Subtract one from another, 
  \begin{equation*}
    [H^+] = [OH^-] + [H_2S] - [S^{2-}]
  \end{equation*}
  Substitute, 
  \begin{equation*}
    [H^+] = \frac{K_w}{[H^+]} + \frac{[HS^-][H^+]}{K_{a1}} - \frac{K_{a2}[HS^-]}{[H^+]}
  \end{equation*}
\end{frame}
\begin{frame}
  \frametitle{Proof}
  Simplify the equation, 
  \begin{equation*}
    [H^+]\left(\frac{[HS^-] - K_{a1}}{K_{a1}}\right) = \frac{1}{[H^+]}\left(K_{a2}[HS^-] - K_w\right)
  \end{equation*}
  Assume that
  \begin{itemize}
    \item $K_{a1}$ and $K_{a2}$ is small such that $[HS^-]\approx c$.
    \item $c >> K_{a1}$ and $c K_{a2} >> K_w$ (or $c >> K_w/K_{a2}$).
  \end{itemize}
  Now, we obtain
  \begin{equation*}
    [H^+]^2 = K_{a1}K_{a2} \Rightarrow pH = \frac{1}{2}(pK_{a1} + pK_{a2})
  \end{equation*}
  This is our motivation for designing buffers.
\end{frame}
\begin{frame}
  \frametitle{Buffer}
  A \textit{buffer} is a solution that resists a change in pH when a small amount of acid or base is added. It is usually a 
  weak acid with its conjugate base, or a weak base with its conjugate acid.
  \blankline
  A buffer is typically made of equal concentrations of both the acid/base and its conjugate base/acid (equimolar).
  Equimolar buffers can resist the pH change the best. You will see that in CHEM2110J next term.
  \blankline
  For weak acid buffer, a quick buffer approximation is given by the Henderson-Hasselbalch equation
  \begin{equation*}
    pH = pK_a + \log\frac{[base]_{init}}{[acid]_{init}}
  \end{equation*} 
  The acid-base ratio is usually between 0.1 and 10. 
\end{frame}
\begin{frame}
  \frametitle{Exercise}
  The $pK_a$ for $HNO_2$ is 3.37. Calculate the pH of a buffer solution that is 0.15 M $HNO_2(aq)$ and 0.20 M $NaNO_2(aq)$. 
  \pause
  \blankline
  Ans: 3.49 \\
  You can directly plug the numbers into the Henderson-Hasselbalch equation. Or, alternatively, 
  you can use 
  \begin{equation*}
    \frac{[NO_2^-][H^+]}{HNO_2} = K_a
  \end{equation*}
  and solve for the concentration of $H^+$.
\end{frame}
\section{Titration}
\begin{frame}
  \frametitle{Titration}
  \textit{Titration} is a technique used to determine the concentration of a given solution. 
  For an strong acid-strong base titration, the pH of the solution changes sharply when the pH approaches 7, which is called
  the \textit{stoichiometric point}. 
  \blankline
  For a weak acid-strong base titration, the stoichiometric point is the point where the moles of base added equals the moles of acid.
  To calculate the pH at the stoichiometric point, you need to first calculate the total volume, and then calculate the pH of the salt 
  solution.
  \blankline
  An \textit{acid-base indicator} is used in titrations. It is a water soluble dye whose color depends on the pH. Usually, indicators themselves
  are weak organic acids. 
\end{frame}
\begin{frame}
  \frametitle{Exercise}
  Calculate the pH at the stoichiometric point of the titration of 25.00 mL of 0.020 M $NH_3(aq)$ with 0.015 M $HCl(aq)$. (For $NH_4$, 
  $K_a = 5.6\times 10^{-10}$.)
  \pause
  \blankline
  Ans: 5.66 \\
  Calculate the moles of $NH_3$, and the volume of $HCl$ needed. Then, calculate the concentration of $NH_4Cl$ at the point. Use $[H^+] \approx \sqrt{cK_a}$ to
  calculate the pH.
\end{frame}
\section{Solubility Equilibria}
\begin{frame}
  \frametitle{The Solubility Product}
  $K_{sp}$, the solubility product, is defined as the equilibrium constant for the solubility equilibrium. You can treat it 
  as a normal equilibrium constant.
  \blankline
  Since different salts have different solubility, we can precipitate unwanted ions using the \textit{common-ion effect}.
  \blankline
  Precipitation occurs when $Q_{sp} > K_{sp}$. We can predict the order of precipitation with $K_{sp}$. Salts with smaller $K_{sp}$
  precipitate first.
  \blankline
  \textit{Coordination complex} is formed when a Lewis acid and a Lewis base react. The formation of $Ag(NH_3)_2^+$ is an example. 
  You are not required to know much about the topic.
\end{frame}
\begin{frame}
  \frametitle{Tips for Your Final}
  This is my last regular RC. Here are some tips for your final exam.
  \begin{itemize}
    \item Chapter 00 - 06 will also be tested. Do review them!
    \item Spare some time for the sample exam. It is a good chance to check your understanding.
    \item Prepare your cheating paper.
    \item Get enough sleep before the exam.
  \end{itemize}
  Good luck! For future TAs of this course, you can refer to this link for the \LaTeX source file of my RC slides
  \href{https://github.com/JeffSnn/CHEM2100J-RC-Slides}{here}(click on it) 
\end{frame}
% insert a reference frame before the 'thank you' frame ----------------------
\begin{frame}
  \frametitle{References}
  
  \begin{thebibliography}{99} % Beamer does not support BibTeX so references must be inserted manually as below
  \bibitem{slides}
  Milias Liu, CHEM2100J slides, Fall 2023.
    
  \bibitem{eiram2013cvssv2}
  Peter Atkins, et. al., \textit{Chemical Principles}, 7th Ed. ISBN: 978-1-4641-8395-9
    
  \end{thebibliography}
\end{frame}

\begin{frame}
  \Huge{\centerline{Thank you!}}
\end{frame}


\end{document}
