%----------------------------------------------------------------------------------------
%	TITLE PAGE
%----------------------------------------------------------------------------------------
\documentclass[aspectratio=169]{beamer}
% use 16:9 in new slides!

\usepackage{tikz}
\usepackage{array}

\usepackage[skins]{tcolorbox}

\def\blankline{\\[6pt]}
\def\bpr#1#2{
\begin{tcolorbox}
[boxsep=.15cm,left=.2cm,right=.2cm,oversize,boxrule=0mm,
colback=green!55!blue!20!white!60,
colframe=red!50!yellow!50!white,
colbacktitle=blue!50, 
coltitle=black,enhanced,drop fuzzy shadow,
fonttitle=\bfseries,title=#1]
#2
\end{tcolorbox}
}

\def\bdef#1#2{
\begin{tcolorbox}
[boxsep=.15cm,left=.2cm,right=.2cm,oversize,boxrule=0mm,
colback=white!60,
colframe=red!50!yellow!50!white,
colbacktitle=green!50!yellow!60!gray, 
coltitle=black,enhanced,drop fuzzy shadow,
fonttitle=\bfseries,title=#1]
#2
\end{tcolorbox}
}

\def\bthe#1#2{
\begin{tcolorbox}
[boxsep=.15cm,left=.2cm,right=.2cm,oversize,boxrule=0mm,
colback=white!60,
colframe=red!50!yellow!50!white,
colbacktitle=red!52, 
coltitle=black,enhanced,drop fuzzy shadow,
fonttitle=\bfseries,title=#1]
#2
\end{tcolorbox}
}

\def\tcb#1{
\begin{tcolorbox}
[boxsep=.15cm,left=.2cm,right=.2cm,oversize,boxrule=0mm,
colback=white!60,
colframe=red!50!yellow!50!white,
colbacktitle=red!50!yellow!50!white, coltitle=black,enhanced,drop fuzzy shadow,]
#1
\end{tcolorbox}
}



\definecolor{UMBlue}{RGB}{5,32,103}
\definecolor{UMYellow}{RGB}{255,216,0}
%\usetheme{Madrid}
\setbeamertemplate{headline}{
  \leavevmode%
  \begin{minipage}{0.75\paperwidth}
  \vspace{1.2ex}\hspace{-0.245\paperwidth}
  \resizebox{\paperwidth}{3ex}{
    \tikz{
      \fill [color=UMBlue] (0,0) rectangle (10, 0.13);
      \fill [color=UMYellow] (0,-0.1) rectangle (10, -0.18);
    }
  }
  \begin{beamercolorbox}[wd=\linewidth,ht=2.5ex,dp=1.125ex]{section}
    \insertsubsectionnavigationhorizontal{\linewidth}{}{Slide \insertpagenumber}
  \end{beamercolorbox}
  \end{minipage}
  \begin{minipage}{0.23\paperwidth}
  \hspace{0.2em}
  \includegraphics[width=\linewidth]{assets/logo.png}
  \end{minipage}
}

\setbeamercolor{title}{fg=UMBlue}
\setbeamercolor{frametitle}{fg=UMBlue}
\setbeamercolor{structure}{fg=UMBlue}

\title[Course number]{VC210 Physical Equilibrium} 
\author[]{Sun Qizhen}
\institute[UMJI-SJTU]
{
	University of Michigan - Shanghai Jiaotong University
	\\\medskip
	Joint Institute
}
%----------------------------------------------------------------------------------------
%	Highlight the title of the current section
%----------------------------------------------------------------------------------------
\AtBeginSection[]
{
  \begin{frame}
    \frametitle{Table of Contents}
    \tableofcontents[currentsection]
  \end{frame}
}



\begin{document}
% insert title page---------------------------
\maketitle
%insert contents------------------------------
\begin{frame}
  \frametitle{Table of Contents}
  \tableofcontents
\end{frame}


\section{Phase Transition}
\begin{frame}
  \frametitle{Phase Transition}
  \begin{figure}
    \centering
    \includegraphics[width=0.55\textwidth]{assets/phase_change.png}
  \end{figure}
  Remember it or copy it to your cheating paper!
\end{frame}
\begin{frame}
  \frametitle{Vapor Pressure}
  Origin of Vapor Pressure: At a fixed temperature, as long as some liquid is present, 
  the vapor exerts a characteristic pressure regardless of the amount of liquid water present.
  \blankline
  At boiling point, the vapor pressure equals the atmospheric pressure. Refer to \href{https://zhuanlan.zhihu.com/p/33133532}{this link}(click on it) 
  for more information.
  \blankline
  Qualitatively, strong intermolecular forces lead to low vapor pressure. That is because the intermolecular
  attractions strongly bond the molecules together and prevent them from escaping.

\end{frame}
\begin{frame}
  \frametitle{Vapor Pressure}
  Clausius-Clapeyron Equation:
  \begin{equation*}
    \ln(\frac{p_2}{p_1}) = \frac{\Delta H_{vap}^{\circ}}{R}\left(\frac{1}{T_1} - \frac{1}{T_2}\right)
  \end{equation*}
  Alternatively, 
  \begin{equation*}
    \ln p = \left(\ln p_0 + \frac{\Delta H_{vap}^{\circ}}{RT_1}\right) - \frac{\Delta H_{vap}^{\circ}}{R}\cdot \frac{1}{T}
  \end{equation*} 
  Corollaries:
  \begin{itemize}
    \item Boiling points increase with pressure.
    \item Vapor pressures increase with temperature.
  \end{itemize}
\end{frame}
\begin{frame}
  \frametitle{Phase Diagram}
  \begin{columns}
    \column{0.5\textwidth}
    \begin{itemize}
      \item Slope of the line on the border indicates relative density. A negative slope for the solid/liquid border line means the liquid is denser.
      \item Phase transformation occurs when $(T, p)$ crosses the border line.
      \item Critical point is the end in the liquid/vapour phase boundary. In the supercritical region, the liquid and vapour phases are indistinguishable.
    \end{itemize}
    \column{0.5\textwidth}
    \begin{figure}
      \centering
      \includegraphics[width=0.65\textwidth]{assets/phase_diagram.png}
    \end{figure}
  \end{columns}
\end{frame}
\begin{frame}
  \frametitle{Binary System}
  Ideally, Raoult's Law applies to each component in the mixture:
  \begin{equation*}
    p_A = x_A \cdot p_{pure,A}
  \end{equation*}
  $x_A = n_A/\sum n$ is the mole fraction. Further, we apply Dalton's law, 
  \begin{equation*}
    p_{tot} = \sum x_i p_i
  \end{equation*}
  Corollary:
  \begin{equation*}
    x_{A, g} = \frac{x_{A, l}\cdot p_{A, pure}}{x_{A, l}\cdot p_{A, pure}+x_{B, l}\cdot p_{B, pure}}
  \end{equation*}
\end{frame}
\begin{frame}
  \frametitle{Exercise}
  The vapor pressure of acetone, $C_3H_6O$, at $7.7^{\circ}C$ is 13.3 kPa, and its enthalpy of vaporization is $29.1 \mathrm{kJ}\cdot \mathrm{mol}^{-1}$. 
  Estimate the normal boiling point of acetone.
  \pause
  \blankline
  Ans: $62.3^{\circ}C$ or $335.3$ K.
  \blankline
  Note that the normal boiling point is the boiling point at 1 atm. We use
  \begin{align*}
    \frac{1}{T} = \frac{1}{T_0} + \frac{R}{\Delta H_{vap}^{\circ}}\ln \frac{p_0}{p}
  \end{align*}
  Plug in the numbers and get the result.
\end{frame}
\begin{frame}
  \frametitle{Exercise}
  Given that the vapor pressures of pure benzene and toluene at $25^{\circ}C$ are 94.6 and 29.1 Torr, respectively. Calculate
  \begin{itemize}
    \item[(a)] The vapor pressure at $25^{\circ}C$ of a solution of benzene in toluene where the mole fraction of benzene is 0.900.
    \item[(b)] The mole fraction of benzene in the vapor. 
  \end{itemize} 
  \pause
  Ans: (a) 88.0 Torr; (b) 0.967. \\
  The vapor pressure is calculated by
  \begin{equation*}
    p_{tot} = x_{b}p_{b} + x_{t}p_{t}
  \end{equation*}
  The mole fraction in the vapor is calculated by
  \begin{equation*}
    x_{b, g} = \frac{x_{b, l}p_{b}}{x_{b, l}p_{b}+x_{t, l}p_{t}}
  \end{equation*}
\end{frame}
\section{Solubility}
\begin{frame}
  \frametitle{Solubility}
  The solubility of a gas: Henry's Law
  \begin{equation*}
    s = k_H p
  \end{equation*}
  Solids dissolve into liquids: 
  \begin{equation*}
    \Delta H = \Delta H_{L} + \Delta H_{hyd} 
  \end{equation*}
  Usually, sublimation ($\Delta H_L$) is endothermic and hydration ($\Delta H_{hyd}$) is exothermic.

  Dissolving a solid leads to a positive $\Delta S_{sys}$. Under high temperature, solids dissolve faster but 
  not necessarily more.
  \blankline
  Liquids dissolve into liquids: like dissolves like.
\end{frame}
\begin{frame}
  \frametitle{Colligative Properties}
  \textit{Colligative properties} are properties of solutions that depend only on the number of solute and solvent particles, not on the nature of the solute particles.
  \blankline
  You should remember the following concepts:
  \begin{itemize}
    \item Mole fraction: $x_i = n_i/n_{tot}$
    \item Molality: $m_i = n(solute)/m(solvent)$
    \item Molarity: $M_i = n(solute)/V(solution)$
  \end{itemize}
  Do not mix them up!
\end{frame}
\begin{frame}
  \frametitle{BP Elevation and FP depression}
  Boiling point elevation ($m$ is molality, $k$ is constant, $i$ is the Van't Hoff factor):
  \begin{equation*}
    \Delta T = i k_b \cdot m
  \end{equation*}
  Freezing point depression:
  \begin{equation*}
    \Delta T = i k_f \cdot m
  \end{equation*}
  FP depression is more significant than BP elevation. For non-electrolyte, $i = 1$. For electrolyte, $i$ is the number of ions.

  Osmosis pressure:
  \begin{equation*}
    \Pi = iRTc
  \end{equation*}
  where $R$ is the ideal gas constant, $T$ is the temperature, and $c$ is the molarity.
\end{frame}
\begin{frame}
  \frametitle{Exercise}
  The osmotic pressure of 1.50 g of polymethyl methacrylate dissolved in enough methylbenzene to produce 
  175 mL of solution was 2.11 kPa at $20^{\circ}C$. Calculate the average molar mass of the sample of polymethyl methacrylate.
  \pause
  \blankline
  Ans: 9.87 kg/mol. \\
  Plug numbers into the following equation.
  \begin{equation*}
    n = cV = \frac{\Pi V}{iRT}
  \end{equation*}
\end{frame}
% insert a reference frame before the 'thank you' frame ----------------------
\begin{frame}
  \frametitle{References}
  
  \begin{thebibliography}{99} % Beamer does not support BibTeX so references must be inserted manually as below
  \bibitem{slides}
  Milias Liu, CHEM2100J slides, Fall 2023.
    
  \bibitem{eiram2013cvssv2}
  Peter Atkins, et. al., \textit{Chemical Principles}, 7th Ed. ISBN: 978-1-4641-8395-9
    
  \end{thebibliography}
  \end{frame}

\begin{frame}
  \Huge{\centerline{Thank you!}}
\end{frame}


\end{document}
